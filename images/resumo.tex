\documentclass[12pt]{article}

% ---------- PACOTES BÁSICOS ----------
\usepackage[utf8]{inputenc}
\usepackage[T1]{fontenc}
\usepackage[brazil]{babel}
\usepackage{geometry}
\usepackage{parskip}
\usepackage{indentfirst}
\usepackage{graphicx}
\usepackage{float}

\geometry{a4paper, margin=2.5cm}
\setlength{\parindent}{1.25cm}

% ---------- DADOS DO DOCUMENTO ----------
\title{\textbf{Capítulo 5: Buscas}}
\author{Vitor Henrique Carvalho de Morais - 12221ECP011}

\begin{document}

\maketitle

% ---------- IDENTIFICAÇÃO (opcional) ----------
\noindent
\textbf{Disciplina: Programação Lógica e Inteligência Artificial} \\ 
\textbf{Professor(a): Marcelo Rodrigues} \\ 
\textbf{Curso: Engenharia de Computação}

\vspace{1cm}

% ---------- INTRODUÇÃO ----------
\section*{Introdução}

Neste capítulo, estudamos como funcionam as buscas na programação lógica, 
especialmente utilizando a linguagem Prolog. A ideia principal é entender 
como um computador pode encontrar soluções para um problema seguindo regras 
bem definidas, quase como se estivesse explorando diferentes caminhos até 
chegar à resposta correta. Em vez de dizer exatamente como o programa deve 
fazer cada passo, na programação lógica nós descrevemos o problema e as regras, 
e o próprio sistema se encarrega de procurar a solução.

Também comparamos o Prolog com outras linguagens de programação. Em linguagens 
mais comuns, como JavaScript, o programador precisa escrever passo a 
passo como a busca será feita. Já no Prolog, o mecanismo de busca faz parte da 
própria linguagem, funcionando automaticamente por meio de regras e tentativas 
sucessivas até encontrar uma solução. Essa comparação ajuda a entender as diferenças 
entre os estilos de programação e mostra como cada abordagem resolve problemas de 
maneira diferente.

% ---------- DESENVOLVIMENTO ----------
\section*{Resumo: Capítulo 5 - Buscas}

\subsection*{5.1 - Árvores de Busca}

Uma árvore de busca é uma forma de organizar todas as possibilidades para resolver um problema.
Imagine que você está tentando encontrar a saída de um labirinto. Você começa em um ponto inicial 
e, a cada passo, pode escolher diferentes caminhos. Cada escolha leva a novas possibilidades, e essas 
possibilidades continuam se dividindo até que você encontre a saída ou perceba que aquele caminho não 
funciona.

A árvore de busca funciona como um desenho dessas escolhas. O ponto onde tudo começa é chamado de raiz. 
A partir dele, surgem vários caminhos, como galhos de uma árvore. Cada novo lugar que você pode chegar 
representa uma nova situação do problema. Quando um caminho não pode mais continuar, ele termina, como 
uma folha da árvore. Assim, a árvore de busca ajuda o computador a explorar diferentes caminhos de forma 
organizada até encontrar a solução correta.

\begin{figure}[H]
    \centering
    \includegraphics[width=0.6\textwidth]{arvore_busca.png}
    \caption{Exemplo de uma árvore de busca.}
    \label{fig:minha_imagem}
\end{figure}

Imagine que essa imagem é como um mapa de escolhas.

Lá em cima está a letra B, que é o ponto onde tudo começa. 
É como se você estivesse em um jogo e tivesse que decidir para onde ir primeiro. 
A partir de B, você tem três caminhos possíveis, representados pelos números 1, 2 e 3. 
Cada número é uma escolha diferente.

Se você escolhe o caminho 1, você chega na letra F. Depois disso, ainda 
pode escolher entre dois novos caminhos, que levam até G ou H. Quando você chega em 
G ou H, o caminho termina, como se fosse o fim daquela tentativa.

Se você escolhe o caminho 2 logo no começo, você chega em D, e ali o caminho também 
termina.

Se você escolhe o caminho 3, você vai para C, e de lá pode seguir para P ou Q, que 
também são finais.

Então essa imagem mostra como, a partir de um começo, várias escolhas vão surgindo, 
como galhos de uma árvore. Cada galho leva a um lugar diferente. A ideia da árvore 
de busca é justamente mostrar todos os caminhos possíveis que você pode seguir até 
chegar em um resultado.

\subsection*{5.2 - Busca em Largura}

Imagine que você está procurando um tesouro escondido em um castelo cheio de portas. 
Em vez de sair correndo por um corredor até o fim, a busca em largura funciona assim: 
você olha primeiro todas as portas que estão bem perto de você. Depois que verifica todas 
elas, você passa para as portas que ficam um pouco mais longe. Só depois você vai para as 
ainda mais distantes. Ou seja, você vai explorando “andar por andar”, sem se aprofundar 
demais em um único caminho. Assim, se o tesouro estiver perto do começo, você encontra 
mais rápido.

\begin{figure}[H]
    \centering
    \includegraphics[width=0.6\textwidth]{busca_largura.png}
    \caption{Exemplo de busca em largura.}
    \label{fig:minha_imagem}
\end{figure}

Para organizar isso, usamos algo chamado fila. Pense em uma fila como a de uma padaria: 
quem entra primeiro é atendido primeiro. Na busca em largura, cada caminho que ainda 
precisa ser explorado entra no final da fila. O caminho que entrou primeiro será o 
primeiro a ser testado.


\begin{figure}[H]
    \centering
    \includegraphics[width=0.4\textwidth]{filas.png}
    \caption{Exemplo de fila para busca em largura.}
    \label{fig:minha_imagem}
\end{figure}

As filas estendidas são simplesmente a fila principal já com os novos caminhos adicionados 
no final. Quando você pega um caminho da frente da fila e descobre novos caminhos possíveis 
a partir dele, você coloca esses novos caminhos no fim da fila. Assim, a fila vai ficando 
maior, cheia de possibilidades organizadas.

\begin{figure}[H]
    \centering
    \includegraphics[width=0.6\textwidth]{fila_extendida.png}
    \caption{Exemplo de fila estendida para busca em largura.}
    \label{fig:minha_imagem}
\end{figure}

Já a fila de extensões é o conjunto desses novos caminhos que acabaram de nascer a partir 
de um caminho que você acabou de explorar. Primeiro você cria essas extensões 
(os novos caminhos possíveis) e depois coloca todas elas no final da fila principal. 
Dessa forma, a busca continua organizada, sempre explorando primeiro os caminhos mais 
antigos antes de passar para os mais recentes.

\begin{figure}[H]
    \centering
    \includegraphics[width=0.6\textwidth]{fila_extensoes.png}
    \caption{Exemplo de fila de extensões para busca em largura.}
    \label{fig:minha_imagem}
\end{figure}

% ---------- CONCLUSÃO ----------
\section*{Conclusão}

Síntese das ideias principais.

% ---------- REFERÊNCIAS (opcional) ----------
\section*{Referências}

Autor. \textit{Título da obra}. Editora, ano.

\end{document}